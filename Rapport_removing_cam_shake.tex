\documentclass[a4paper,10pt]{report}
\usepackage[utf8]{inputenc}

% Title Page
\title{Implémentation d'article de recherche \\ 
Removing Camera Shake via Weighted Fourier Burst Accumulation}
\author{Sofiane Horache, Roman Fenioux}
\date{April 2017}


\begin{document}
\maketitle

% pour éviter d'avoir la numérotation 0.1 0.2 etc avant les sections
\setcounter{secnumdepth}{0}

% 1st page, résumé de l'article
\section{Introduction}
\paragraph{}
Le problème adressé ici est le flou de bougé, présent notamment dans les photographies 
amateur prises à la main dans des environnements mal éclairés. Pour avoir une photographie 
suffisament exposée il est souvent necessaire d'augmenter le temps d'exposition lorsqu'on 
ne peut ou ne veut pas agir sur l'ouverture ou la sensibilité de l'appareil. Mais si la 
position de l'appareil change durant l'acquisition, on voit apparaître du flou dans la direction
du mouvement. 

\section{Approche de l'article}
\paragraph{}
Le mouvement de l'appareil durant la prise de vue peut-être modélisé par une convolution de l'image
par un noyau de flou. la direction du mouvement est aléatoire d'une image à l'autre et on veut donc 
récupérer l'information que contient chaque image pour obtenir une image nette. L'approche utilisée 
ici consiste à prendre une rafale d'images au lieu d'une seule, puis à les combiner pour éliminer ce 
flou. Cette technique permet même de débruiter l'image. Pour une séquence de M images on a :
\[
  v_{i} = u_{i} \ast k_{i} + n{i}
\]
Pour ce faire l'article propose de calculer une moyenne pondérée de leurs transformées 
de Fourier pour éviter d'avoir à faire une estimation explicite du noyau de flou.
\[
  u_{p}(x) = \mathcal{F}^{-1}\left(\sum_{i=1}^M w_{i}(\zeta)\cdot \hat{v_{i}}(\zeta)\right)(x)
\]
\[
  w_{i}(\zeta) = \frac{|\hat{v_{i}}(\zeta)|^p}{\sum_{j=1}^M |\hat{v_{j}}(\zeta)|^p}
\]

\section{Notre implémentation}
\paragraph{}
A compléter : explications sur ce qu'on a fait

\section{Nos résultats}
\paragraph{}
A compléter : illustrations de nos tests et résultats

\section{Analyse et Commentaires}
\paragraph{}
A compléter : analyse de la méthode, points forts / faibles, limites et améliorations possibles
Pb : recalage d'images floues.. ? tester les limites

\end{document}          
